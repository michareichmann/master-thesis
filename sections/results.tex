This thesis could demonstrate the functioning of COCPITT with all its desired features. It was proven that is possible to read out the analogue CMS pixel chips with a \ac{DTB}. All arising problems could be solved. Due to the digital readout there is in principle no limit for the data-taking any more except the size of a computer's hard drive.\\
It was also shown that the telescope is very flexible and easy to deploy in various environments. It can be used to examine different \ac{DUT}s like diamond pad detectors or digital CMS pixel detectors for example.\\
Furthermore, the telescope delivers a reliable trigger for itself and for the \ac{DUT} that has a freely adjustable trigger area using the ability to mask pixels. Above all it has been demonstrated that COCPITT gathers meaningful information about pulse heights and particle tracks.\\
The goal for the future is a complete table-top set-up of the telescope by replacing replacing the trigger logic by the \ac{TU} of the Ohio State University. That way the telescope will work out of the box with almost no installation time. It is also planned to conduct a study of telescope's resolution exploiting the residual distribution of the particle tracks.
% ========================================================
