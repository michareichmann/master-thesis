% \tableofcontents
% ========================================================
% INTRO
% ========================================================
Synchronously with the Large Hadron Collider reaching higher and higher centre of mass energies and luminosities a constant evolution of the particle detectors is inevitable. The detectors have to cope with increasing particle fluxes and particles with higher energies. COCPITT, the COmpaCt Particle Tracking Telescope, was built in order to push forward and support the development of new technologies for particle detectors. Using telescopes to accomplish that goal is a well established procedure as proven by the success of the various EUDET telescopes like AIDA or DATURA. Though telescope literally means ``far-seeing'' the name origins by the utilisation of several planes to gather information about a device under test.\\
COCPITT originates from the Institute of Particle Physic at ETH Z{\"u}rich where it was designed and mounted. The aims of the telescope are high resolution tracks  exploiting single CMS pixel detectors, providing reliable triggers at high rate in the MHz range, and deliver accurate pulse height information. Its design is mainly driven by two facts: Cost efficiency while minimising the amount of material and a good flexibility in terms of transportation and installation. The construction was carried out in two stages. A first version was built to test planes of CMS' Pixel Luminosity Telescope. Based on the experience of this version a second version was produced in order to test the new digital pixel chip for CMS.\\
The purpose of this master thesis is first and foremost the commissioning of COCPITT, to ascertain its functioning and guarantee a stable and reliable readout. Furthermore, insights of the telescope performance and the investigation of various devices under test will be provided. During the time of this thesis, the telescope was used to gather information about diamond pad and diamond pixel detectors as well as well as about the digital CMS pixel detector. The experiments to collect this data were performed during different beam tests at DESY, PSI and CERN. Using the experience of the beam tests, the installation and the utilisation of COCPITT were successively improved.
