% BEGIN
% ETH STYLE -> DON'T CHANGE
\documentclass[british,11pt,a4paper]{memoir}
\usepackage{xltxtra}
\usepackage[utf8]{inputenc}
\usepackage[OT1]{fontenc}
\usepackage{babel}
\usepackage[sc]{mathpazo}
\usepackage{amsmath,amssymb,amsfonts,mathrsfs}
\usepackage[amsmath,thmmarks]{ntheorem}
% =======================================================
\usepackage{soul}
\usepackage{pdfpages}
\graphicspath{ {Pics/} }
\input{template/extrapackages}
\def\labelitemii{\textopenbullet}  % sets the symbols in the itemize environment
\def\labelitemiii{$\triangleright$}
\newcommand{\no}{\noindent}
\newcommand{\as}{\\[14pt]}
\newcommand{\s}{\\[7pt]}
\newcommand{\ka}{\hspace*{0.5cm}}
\newcommand{\ma}{\hspace*{1cm}}
\newcommand{\ga}{\hspace*{1.5cm}}
\newcommand{\li}{\left|}
\newcommand{\re}{\right|}
\newcommand{\lii}{\left\langle}
\newcommand{\ree}{\right\rangle}
\newcommand{\lka}{\left(}
\newcommand{\rkz}{\right)}
\newcommand{\intsum}{\ensuremath{\int\hspace{-17pt}\sum}}
\newcommand{\intsumm}{\ensuremath{{\int}\hspace{-12pt}\sum}}
\newcommand{\const}{\text{const.}}
\newcommand{\z}{\text}
\newcommand{\h}{\hslash}
\newcommand{\ar}{\autoref}
\newcommand{\fa}{\hspace{-4pt}\downarrow}
\newcommand{\wf}{\hspace{-4pt}\uparrow}
\newcommand{\cc}{\cdot}
\newcommand{\eps}{\upvarepsilon}
\newcommand{\lagr}{\mathcal{L}}
\newcommand{\lagri}{\mathcal{L}\z{I}}
\newcommand{\lagrii}{\mathcal{L}\z{II}}
\newcommand{\ham}{\mathcal{H}}
\newcommand{\bul}{\item[\textopenbullet]}
\newcommand{\terminal}[1]{\colorbox{black}{\textcolor{white}{{\fontfamily{phv}\selectfont \large{#1}}}}}
\input{template/layoutsetup}
\input{template/theoremsetup}
\input{template/macrosetup}
\usepackage[linkcolor=black,colorlinks=true,citecolor=black,filecolor=black]{hyperref}
\providecommand\subfigureautorefname{Figure}
\newsubfloat{figure}
\makeindex
% END
\begin{document}
\chapter{The Reconnaissance of the Unknown - Experiments \& Measurements}
% ========================================================
% DIGITAL READOUT
% ========================================================
\section{Digital Readout of the Analogue Chip}
Initially the analogue chips were read out with an \ac{ATB} and a software called ``psi46expert''. As already mentioned in \ar{sdtb} the \ac{DTB} has some advantages compared to the \ac{ATB}, which was the reason for trying to readout the analogue chips with the \ac{DTB} and pXar. In order to be able to accomplish that, Simon Spannagel wrote an extension to the pXar core that is able to read the data of the analogue \ac{ROC}. The extension includes the decoder described in \ar{sdec} and adds all missing \ac{DAC}s that are not used for the digital chip any more to the library. In addition it utilized the connection to the \ac{ADC} of the \ac{DTB}.
% ========================================================
\subsection{Functioning Check}
First of all the basic functioning should be checked. Before connecting to the \ac{DTB} it is recommended to ascertain that the \ac{DUT} is working\footnote{Working in this context means, that it is possible to successfully readout calibrate signals} and that is has a set of appropriate \ac{DAC}s for analogue \ac{ROC} (e.g. with the \ac{ATB}). For the test board parameter file it is best to use the following lines:\s
{\ubuntumono 
	0		clk   4\\
	1		ctr   4\\
	2		sda   19\\
	3		tin   9\s}
For the other necessary files it possible to get default files (q.v. pXar twiki).

% ========================================================
\subsection{\ac{DTB} Timing}
% ========================================================
\subsection{Sampling Point of the \ac{ADC}}
% ========================================================
% WBC Scan
% ========================================================
\section{WBC Scan}
% ========================================================
% FAST-OR
% ========================================================
\section{Fast-OR Dependencies}
% ========================================================
% TRIMMING
% ========================================================
\section{Trimming}
% ========================================================
% DYING FAST-OR
% ========================================================
\section{\ac{TBM} problems - dying fast-OR}
% ========================================================
% CLI TESTS
% ========================================================
\section{\ac{CLI} Test Implementations}
% ========================================================
% DIA SHADOW
% ========================================================
\section{Finding \ac{DUT} Shadows}
% ========================================================
% EUDAQ AND TLU
% ========================================================
\section{Implementation of EUDAQ and \ac{TLU} }
% ========================================================
% PULSE HEIGHT CAL
% ========================================================
\section{Pulse Height Calibration}
% ========================================================
% TO GET IT COMPILED
% ========================================================
\chapter*{List of Acronyms}
\input{acronyms.dat}
\bibliographystyle{plain}
% \bibliography{refs}
\end{document}
