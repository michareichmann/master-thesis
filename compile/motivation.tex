\documentclass[british,11pt,a4paper]{memoir}
\input{preamble}
\begin{document}
\chapter{The Origin of the Adventure - Motivation}
% ========================================================
Synchronously with the Large Hadron Collider reaching higher and higher centre of mass energies and luminosities a constant evolution of the particle detectors is inevitable. The detectors have to cope with increasing particle fluxes. COCPITT, the COmpaCt Particle Tracking Telescope, was built in order to push forward and support the development of new technologies for particle tracking detectors. Using telescopes to accomplish that goal is a well established procedure as proven by the success of the various EUDET telescopes like AIDA or DATURA. Though telescope literally means ``far-seeing'' the name stems from the utilisation of several planes to gather information about a device under test. Each single telescope plane is set up to send a trigger if it detects a passing particle and delivers two dimensional positioning and pulse height data about these particles.\\
COCPITT originates from the Institute of Particle Physic at ETH Z{\"u}rich where it was designed and assembled. The aim of the telescope is to provide reliable triggers at high rate in the order of $10\,$MHz and additionally contribute high resolution tracks in the sub-millimetre range and pulse height information. In order to accomplish that single analogue CMS pixel detectors are exploited. Its design is mainly driven by two facts: Simplicity by means of using already available components and a good flexibility in terms of transportation and installation. The construction was carried out in two stages. A first version was built to test planes of CMS' Pixel Luminosity Telescope. Based on the experience of this a second version was produced in order to test the new digital pixel chip for CMS, which resulted in a change of the readout from analogue to digital.\\
The purpose of this master thesis is first and foremost to inform about the commissioning of COCPITT, how its functioning was ascertained and a stable and reliable readout was achieved. Furthermore, insights of the telescope performance and the investigation of various devices under test will be provided. During the time of this thesis, the telescope was used to gather information about diamond pad and diamond pixel detectors as well as about the digital CMS pixel detector. The experiments to collect this data were performed during different beam tests at DESY, PSI and CERN. Using the experience of the beam tests, the installation process and the functioning of COCPITT were successively improved.
% ========================================================
\end{document}