% BEGIN
% ETH STYLE -> DON'T CHANGE
\documentclass[british,11pt,a4paper]{memoir}
\usepackage[utf8]{inputenc}
\usepackage[OT1]{fontenc}
\usepackage{babel}
\usepackage[sc]{mathpazo}
\usepackage{amsmath,amssymb,amsfonts,mathrsfs}
\usepackage[amsmath,thmmarks]{ntheorem}
% =======================================================
\usepackage{soul}
\usepackage{pdfpages}
\graphicspath{ {Pics/} }
\input{template/extrapackages}
\def\labelitemii{\textopenbullet}  % sets the symbols in the itemize environment
\def\labelitemiii{$\triangleright$}
\newcommand{\no}{\noindent}
\newcommand{\as}{\\[14pt]}
\newcommand{\s}{\\[7pt]}
\newcommand{\ka}{\hspace*{0.5cm}}
\newcommand{\ma}{\hspace*{1cm}}
\newcommand{\ga}{\hspace*{1.5cm}}
\newcommand{\li}{\left|}
\newcommand{\re}{\right|}
\newcommand{\lii}{\left\langle}
\newcommand{\ree}{\right\rangle}
\newcommand{\lka}{\left(}
\newcommand{\rkz}{\right)}
\newcommand{\intsum}{\ensuremath{\int\hspace{-17pt}\sum}}
\newcommand{\intsumm}{\ensuremath{{\int}\hspace{-12pt}\sum}}
\newcommand{\const}{\text{const.}}
\newcommand{\z}{\text}
\newcommand{\h}{\hslash}
\newcommand{\ar}{\autoref}
\newcommand{\fa}{\hspace{-4pt}\downarrow}
\newcommand{\wf}{\hspace{-4pt}\uparrow}
\newcommand{\cc}{\cdot}
\newcommand{\eps}{\upvarepsilon}
\newcommand{\lagr}{\mathcal{L}}
\newcommand{\lagri}{\mathcal{L}\z{I}}
\newcommand{\lagrii}{\mathcal{L}\z{II}}
\newcommand{\ham}{\mathcal{H}}
\newcommand{\bul}{\item[\textopenbullet]}
\newcommand{\terminal}[1]{\colorbox{black}{\textcolor{white}{{\fontfamily{phv}\selectfont \large{#1}}}}}
\input{template/layoutsetup}
\input{template/theoremsetup}
\input{template/macrosetup}
\usepackage[linkcolor=black,colorlinks=true,citecolor=black,filecolor=black]{hyperref}
\makeindex
% END
\begin{document}
\chapter{The Doorway to the Information - Data Acquisition}
% ========================================================
% INTRO
% ========================================================
\section{Introduction}
This chapter explains how the interfacing between a computer and the telescope works, in which way the \ac{ROC} can be programmed and how the data is acquired and decoded by the software.\\
All communication between the telescope and a computer is done via the core libraries of a program called pXar, short for Pixel eXpert Analysis Readout. It was written to operate the digital \ac{ROC}. The program is completely composed in C++ and the main part, the pXar-core, consists of a \ac{HAL} and an \ac{API}. Using the basic functions from the pXar-core library there are already a variety of tests that can be performed via the pXar-\ac{GUI}. For easier access and better flexibility there is also a cython translation of the main pXar-core functions to make them accessible via python. This has the huge advantage to easily write own tests and use them via a python \ac{CLI}. To save the data of a single telescope one could also use the pXar software, but since we are interested in the data combining with other instruments, e.g. pad detectors or second telescope, it is much more convenient to use the EUDAQ software core functionalities.\\
Where the software can be found is listed in \ar{ssoft}
% ========================================================
% PXAR
% ========================================================
\section{pXar}
% \includegraphics[width=4cm]{pxar_logo}
pXar includes more than the main functions of the pXar-core that will be presented, but these are the components everything else is built on. An overview of the full software architecture is shown in \ar{p13}.\\
The NIOS II soft core \ac{CPU} is a synthetic \ac{CPU} that is embedded in the \ac{FPGA} of the \ac{DTB}. It runs at $50\,$MHz and is used to perform simple commands like starting the pattern generator. It's main purpose is the control of the \ac{DTB} functionality. Since the correct commands are automatically picked by the \ac{API} of the pXar-core, it is not further investigated. For additional information look at \cite{spannagel}.
\begin{figure}[ht]
	\includegraphics[width=0.95\textwidth]{pxar_scheme}
	\caption{Software architecture of pXar \cite{spannagel}}
	\label{p13}
\end{figure}
\subsection{pXar Core}
% ========================================================
\subsubsection{Hardware Abstraction Layer}
The \ac{HAL} of the pXar-core is responsible for the direct communication with the \ac{DTB}. This is the part that has access to the \ac{RPC} and USB. It supervises the data readout using pipes, which is a buffered readout of the data from the test board and is shown in \ar{preadout}.
\begin{figure}[ht]
	\vspace*{-5pt}
	\centering
	\includegraphics[width=10cm]{hal_buf}
	\caption{Schematic of the data readout.}
	\label{preadout}
	\vspace*{-5pt}
\end{figure}\no
The pipes can be set up for different purposes, either to deliver fully decoded events or ``raw'' events that are only split by the \ac{TBM} header and trailer or complete raw data from the \ac{DTB}. The second option is very useful to debug the system, because it gives insights to the raw, undecoded digital data. The last option may be useful for debugging too, but is mainly used for saving the raw data to a binary file.\\
Most of the basic functions and a short explanation are listed in \ar{t5}. They are loosely divided in categories and all of them can be used while the \ac{DTB} and pXar are running. So, having them in the group ``Device Initialisation'', means that they are used while initialising the board even though the may be reused afterwards.
\begin{table}[ht]
	\begin{tabularx}{\textwidth}{L{4.5cm}|X}
		\noalign{\hrule height 2pt}
		\multicolumn{2}{c}{\textbf{Device Initialisation}}							\\\noalign{\hrule height 2pt}
		\multicolumn{1}{c}{\textbf{command}}		& 	\multicolumn{1}{c}{\textbf{purpose}}																				\\\hline
		initTestboard [signal delay] [pg setup]		& initialises the \ac{DTB} with its device specific \textit{signal delay} settings and the sets up the \ac{PG}			\\
		setTestboardPower [va] [vd] [ia] [id]		& sets the \textit{analogue} and \textit{digital voltage} and \textit{current limits}									\\
		setTestboardDelays [signal delays]			& sets \textit{signal delay} settings																					\\
		flashTestboard [flashfile]					& flashes a given \textit{firmware file} to the \ac{FPGA}																\\
		initROC	[\ac{I2C}] [\ac{DAC}s]				& initialises a \ac{ROC}s of a given \textit{type} with \textit{\ac{I2C} address} with a \textit{vector of \ac{DAC}s} 	\\
		SetupPatternGenerator [pg setup]			& sets the \ac{PG} to a given sequence																					\\
		SetClockSource [source]						& sets the clock source to internal or external																			\\
		\noalign{\hrule height 2pt}
	\end{tabularx}
	\caption{Device initialisation commands of \ac{HAL}. In squared brackets are the arguments of the function.}
	\label{t5}
\end{table}\no
% ========================
\begin{table}[ht]
	\begin{tabularx}{\textwidth}{L{4.5cm}|X}
		\noalign{\hrule height 2pt}
		\multicolumn{2}{c}{\textbf{\ac{DTB} Commands}}							\\\noalign{\hrule height 2pt}
		\multicolumn{1}{c}{\textbf{command}}		& 	\multicolumn{1}{c}{\textbf{purpose}}									\\\hline
		getTBia										& reads analogue current from \ac{DTB}										\\
		getTBva										& reads analogue voltage													\\
		getTBid										& reads digital current														\\
		getTBvd										& reads digital voltage														\\
		set* [limit]								& sets the limit of *$=$TBia, TBva, TBid, TBvd								\\
		SignalProbe* [signal]						& lays one of the internal \textit{signals} to the output *$=$D1, D2, A1, A2\\
		HVoff										& disables the \ac{HV} from the \ac{DTB} to the \ac{ROC}					\\
		HVon										& enables \ac{HV}															\\
		Pon 										& turns on power of the \ac{DTB}											\\
		Poff										& turns off power															\\
		rocSetDAC [\ac{I2C}] [\ac{DAC}] [value]		& sets the \textit{\ac{DAC}} with \textit{\ac{I2C}} to a given \textit{value} \\
		\noalign{\hrule height 2pt}
	\end{tabularx}
	\caption{\ac{DTB} commands commands of \ac{HAL}.}
	\label{t6}
\end{table}\no
% ========================
\begin{table}[ht]
	\begin{tabularx}{\textwidth}{L{4.5cm}|X}
		\noalign{\hrule height 2pt}
		\multicolumn{2}{c}{\textbf{\ac{DAQ} Functions}}											\\\noalign{\hrule height 2pt}
		\multicolumn{1}{c}{\textbf{command}}	& 	\multicolumn{1}{c}{\textbf{purpose}}		\\\hline
		daqStart					& starts a new \ac{DAQ} session								\\
		daqTriggerSource [source]	& sets the trigger source (e.g. extern, pg)					\\
		daqTrigger [number]	[period]& sends out a given \textit{number} of \ac{PG} sequences with \textit{period} 		\\
		daqStop						& stop the current \ac{DAQ} session							\\
		daqEvent					& reads out a decoded event from the buffer if \ac{DAQ} is activated		\\
		daqRawEvent					& reads out a raw event										\\
		daqAllRawEvents				& reads out all events in the buffer						\\
		daqClear					& clears the buffer											\\
		\noalign{\hrule height 2pt}
	\end{tabularx}
	\caption{\ac{DAQ} commands of \ac{HAL}.}
	\label{t7}
\end{table}
% ========================
\begin{table}[ht]
	\begin{tabularx}{\textwidth}{l|X}
		\noalign{\hrule height 2pt}
		\multicolumn{2}{c}{\textbf{\ac{ROC} Functions}}																		\\\noalign{\hrule height 2pt}
		\multicolumn{1}{c}{\textbf{command}}	& 	\multicolumn{1}{c}{\textbf{purpose}}									\\\hline
		SetupI2CValues [\ac{I2C}s]				& sets the available \textit{\ac{I2C}} addresses							\\
		SetupTrimValues	[\ac{I2C}] [trim]		& sets the trim bits from a given \textit{vector} 							\\
		RocSetMask [\ac{I2C}] [mask]			& masks all pixels given in a \textit{vector} 								\\
		PixelSetCalibrate [\ac{I2C}] [col] [row]& enables the calibrate signal for a pixel with \textit{col}, \textit{row} 	\\
		RocClearCalibrate [\ac{I2C}]			& disables the calibrates 													\\
		\noalign{\hrule height 2pt}
	\end{tabularx}
	\caption{\ac{ROC} commands of \ac{HAL}.}
	\label{t8}
\end{table}
% ========================================================
\subsubsection{Application Programming Interface}
The \ac{API} is the interface to the user, this is where all the commands from the \ac{CLI} are directed to. It pre-processing the the data the user enters, i.e. it does assertions, range- and sanity checks for most the input parameters and commands. Thereby it is more user-friendly and misusage can be avoided. To do that, it picks up the commands and functions from \ac{HAL}, the most important ones are listed in \ar{t9}\\
Included inside the \ac{API} is the self-contained \ac{DUT} class, which is responsible for the configuration of the \ac{ROC}. It also saves the configuration settings, which can be read any time. The \ac{DUT} itself is initialised by the \ac{API} function ``initDUT'' and it is compatible with single \ac{ROC}s of both types as well as multi-\ac{ROC} assemblies. Mandatory information to start the initialisation is a file with a list of \ac{DAC} names and their values and a file with trim bit values for all $4160$ pixels for each connected \ac{ROC}.
\begin{table}[ht]
	\begin{tabularx}{\textwidth}{L{4.5cm}|X}
		\noalign{\hrule height 2pt}
		\multicolumn{2}{c}{\textbf{main \ac{API} functions}}						\\\noalign{\hrule height 2pt}
		\multicolumn{1}{c}{\textbf{command}}	& 	\multicolumn{1}{c}{\textbf{purpose}}	\\\hline
		initTestboard [sig delays] [power settings] [pg setup]	& checks weather power setting, \ac{DTB} delays and \ac{PG} settings are viable and then start \ac{HAL} command			\\
		setDAC [\ac{DAC}] [value] [\ac{I2C}]					& verifies that the \ac{DAC} setting are sane and then starts rocSetDAC		\\
		setExternalClock [enable]								& checks weather there is an external clock present and uses SetClockSource accordingly		\\
		daqStart												& uses daqClear, resets the mask, calibrate enable and trim settings, starts \ac{DAQ} with daqStart	from \ac{HAL}			\\
		daqStatus												& returns the status of the \ac{DAQ} 			\\
		getNEnabledPixels										& returns the number of pixels enabled for calibrates			\\
		getNMaskedPixels										& returns the number of masked pixels			\\
		getRocType												& return the \ac{ROC}-type 			\\
		testPixel												& enables the pixel with \textit{column}, \textit{row} and \textit{\ac{I2C}} and check if the pixel exists for the given \ac{ROC}			\\
		maskPixel												& masks the pixel with \textit{column}, \textit{row} and \textit{\ac{I2C}} and check if the pixel exists for the given \ac{ROC}	 			\\
		\noalign{\hrule height 2pt}
	\end{tabularx}
	\caption{Base commands of \ac{API}.}
	\label{t9}
\end{table}
% ========================================================
% 2
% ========================================================
\section{Decoder}\label{sdec}
The decoder of pXar translates the incoming raw data from the \ac{DTB} into meaningful information about the pixel address and pulse height. There are different decoders for the analogue and the digital chips. The decoding for the digital chip is working out of the box since pXar was developed for it. Thus, only the analogue decoder is explained. A description of the basic steps how the raw data from the analogue chip is decoded with pXar is given in the following.\\
After being processed by the \ac{ADC} of the \ac{DTB} and the soft \ac{TBM} the raw data looks like this:
\begin{center}
\terminal{[36600, 4027, 29, 3980, 4083, 39, 90, 141, 16427]}                                                    
\end{center}
The raw data consists of $16\,$bit data words of which only the last $12\,$bits contain the data. The $12\,$bit also correspond to the output range of \ac{DTB}'s \ac{ADC}. The first four bits are used to mark the beginning and the end of an event. So the first step is to reduce the first four bits with a bitwise AND which sets the first four bits to zero:
\begin{equation}
	\z{dataword} = \z{dataword}\ \&\ 0\z{x}0\z{fff}\ \footnote{the prefix ``0x'' heralds a hexdecimal number} \equiv \z{dataword}\ \&\ \left( 2^{12} - 1\right)
\end{equation}
\begin{center}
\terminal{[3832, 4027, 29, 3980, 4083, 39, 90, 141, 43]}                                                    
\end{center}
Because the \ac{ADC} only gives out positive integers, the positive $12\,$bit range is split into two parts where the first $2^{11}$ values correspond to positive values and the second half to the negative ones. So to convert them to their negative equivalent one has to subtract the integers larger than $2^{11}$ by $2^{12}$:
\begin{equation}
	\begin{split}
		&\z{if } \z{dataword}>0\z{x}0800:\\
		&\ \ \ \ \z{dataword} = \z{dataword}\ - 0\z{x}1000 \equiv \z{dataword} - 4096
	\end{split}
\end{equation}
\begin{center}
\terminal{[-264, -69, 29, -116, -13, 39, 90, 141, 43]}                                                    
\end{center}\no\par
\begin{figure}[ht]
	\vspace*{-10pt}
	\centering
	\includegraphics[width=0.95\textwidth]{decode}
	\caption{Schematic of the decoding. The red lines correspond to the ideal position of the level derived from the \ac{UB} and \ac{B}. If a measured value lies in between two dashed grey lines it will be decoded as that level.}
	\label{p15}
\end{figure}
Now the data is transformed into values that correspond to the analogue signal. As overview, the schematic features of the decoding described in the following text is shown in \ar{p15}.\\
Due to various effects (e.g. different supply voltages and \ac{DAC} settings) the absolute values of the levels may vary in the whole range of the \ac{ADC}. But their spacing to one another and their offset is determined by the \ac{UB} and \ac{B} level of the \ac{ROC} header. As already mentioned in \ar{sdata}, the data stream always starts with the \ac{UB} followed by the \ac{B} level of first \ac{ROC}. Even if there are more \ac{ROC} headers to come, the decoding will always take the first header in line as a reference. The distance of two address levels and their maximal deviation are defined by:
\begin{align}
	\z{level}1 &= \frac{\left( \z{\ac{B}} - \z{\ac{UB}} \right)}{4}\\
	\z{levelS} &= \frac{\left( \z{\ac{B}} - \z{\ac{UB}} \right)}{8}
\end{align}
Division in this context always refers to a integer number division without remainder. Level$0$ is always defined by the value of the \ac{B} level. For the example that generates the following values:
\begin{align*}
	\z{level}1 &= \frac{-69+264}{4} = 48\\
	\z{levelS} &= \frac{-69+264}{8} = 24\\
	\z{level}0 &= -69
\end{align*}
For more than one hit pXar is always averaging \ac{UB} and \ac{B}, using a shifting window, to avoid decoding errors due to random deviations.\\
To calculate the address levels the values are first corrected for an offset by subtracting level$0$, which corresponds to the base line. After that the maximum deviation is added and the sum of both is divided by the spacing of two levels (level1).
\begin{equation}
	\z{level} = \frac{\z{dataword} - \z{level}0 + \z{levelS}}{\z{level}1} + 1
\end{equation}
The value of $1$ is added to make all levels positive. Through this operation only values that lie in the region -levelS/+levelS$-1$ around the middle of a spacing will be decoded correctly (q.v. \ar{p15}). The address levels of the example are $[-1, 1, 2, 3, 4]$:
\begin{center}
\terminal{[-264, -69, 29, -1, 1, 2, 3, 4, 43]}                                                    
\end{center}
Once the address levels are extracted the conversion to the pixel address follows the simple equations using the name scheme from \ar{p7}:
\begin{align}
	\z{column} &= 	2\cdot\left(  6\z{C}0 + \z{C}1 \right) + \z{CR}\hspace*{-8pt}\mod_{2}\\
	\z{row} &= 80 - \left( 3\cdot\left(  6\z{R}0 + \z{R}1 \right) + \frac{\z{CR}}{2}\right)
\end{align}
So the example corresponds to the hit of a single pixel with the address $5/12$ (column $5$ row $12$).
% ========================================================
% 3
% ========================================================
\section{Pattern Generator}
The \ac{PG} is able to send out four commands to the \ac{ROC}: a trigger, a  
\begin{wrapfigure}{r}{3.3cm}
	\includegraphics[width=3.2cm]{PG}
	\caption{example sequence of the \ac{PG}, CC stands for clock cycle}
	\label{p14}
\end{wrapfigure} 
token, a calibrate or a reset. A calibrate will sent out a calibration pulse to the \ac{ROC} and the reset basically deletes all hit information on the \ac{ROC}. The pattern generator has a register bank with $256$ addresses at its disposal and each of these addresses can hold a single of the four commands and a delay. In that fashion there may be created arbitrary sequences of the four commands. Once the delay is set to $0$ the \ac{PG} will recognise this as a stop sign an ignore the rest of the sequence. An example is shown in \ar{p14}. This would be a basic set-up to test the functionality of the pixels. After a reset to delete all former hit information a calibrate is sent out. To read out the this event, the delay between calibrate and trigger has to be set to \textit{wbc}$+6$ to compensate for the duration the trigger needs to get to the \ac{ROC}. As the final step, the token is send out to collect the data.
% ========================================================
% 4
% ========================================================
\section{Python Interface}
Using the universal programming language cython, most of the pXar-core functions are also accessible via python. All the available functions are listed in the cython header file ``PyPxarCore.pxd''. Should a function be amiss, it can be added to the header and the source file ``PyPxarCore.pyx`` just like in C++. So that the changes can come into effect and to use the PyPxarCore file the whole program has to be recompiled.\\
Once that is done, the already existing python \ac{CLI} interface ''cmdline.sh`` can be started, which initialises the \ac{DUT} in the same fashion as pXar would do it and allows to program and readout the \ac{DUT} with \ac{API}'s functions. During this thesis this \ac{CLI} was extended by various tests and additional functions. Some of these functions are explained in \ar{sexp}. The extended version is called ''CLIX`` for Command Line Interface eXtension and is started with ''CLIX.sh``.
% ========================================================
% 4
% ========================================================
\section{Running the Software}\label{srunpx}
Once the software is properly installed, meeting all the prerequisites, pXar needs a few configuration files to get started. For each \ac{ROC} that is connected to the \ac{DTB} pXar requires a file called ''dacParameters\_C$<$\ac{I2C}$>$.dat`` with all available \ac{DAC}s and corresponding start values for the respective \ac{ROC} and a file ''trimParameters\_$<$\ac{I2C}$>$.dat`` with a trim bit value for each pixel. If a completely untrimmed \ac{ROC} is required, all values have to be set to $15$ (no trimming). Furthermore a file with delay settings for the \ac{DTB} called ''tbParameters.dat`` and a general configuration file ''configParameters.dat`` are required. Examples of these can be found in the appendix \ar{sconfig} or may be generated with the Perl script ''mkConfig`` in the pXar main directory (q.v. pXar twiki).\\
All the above files are mandatory to start the software and have to be in the same location (configuration directory). The binary files of pXar are in the directory bin of the installation folder. To run the software one has to go there and enter the command:
\ubu{./pXar -d $<$configuration directory$>$}
There are some further useful options to start the program that can be displayed using the ''-h`` option, but it will not start without ''-d``. To name the most important:
\begin{itemize}
	\tri {\ubuntu -f $<$flashfile$>$} \ka $\rightarrow$ flash a firmware from the flashfile to the \ac{FPGA} of the \ac{DTB}
	\tri {\ubuntu -g} \ka $\rightarrow$ directly starts the \ac{GUI}
\end{itemize}
Since the software will start in a rather undocumented \ac{CLI} if started without the \ac{GUI} option, it is recommended starting it always with the ``-g'' option as well.\\
During this thesis only the more user friendly python \ac{CLI} was used, which can be found in the python folder of the installation directory and starts with:

\begin{itemize}\ubuntu
	\tri ./cmdline.sh -d $<$configuration directory$>$
	\tri ./CLIX.sh -d $<$configuration directory$>$
\end{itemize}
and is better for developing new tests, debugging and directly using the \ac{API} commands.
% ========================================================
% 5
% ========================================================
\section{EUDAQ}
EUDAQ is a \ac{DAQ} framework that was designed to be portable, modular and cross platform compatible and was written in C++. It was mainly developed for the EUDET Pixel Telescope, but is very useful for other systems as well \cite{eudaq}. The program is split into different sub processes that can all run on different machines using \ac{TCP} sockets, which are shown in \ar{p16}.\\
The \textit{Run Control} behaves like a central supervisor of the \ac{DAQ} and shows all the other connected processes, which is why it is the only process that has to run. All the hardware that is connected, e.g. the telescope, a \ac{TLU} or other \ac{DUT}s, will have a \textit{Producer} process and are thereby able to be configured, read out and to send the data to the \textit{Data Collector}. The Data Collector collects the data from all \textit{Producers}, combines it to a single data stream and saves it to native binary format. To keep track of potential errors there is also a \textit{Logger} available that receives log messages from all processes, may receive messages from the user via the \textit{Run Control} as well and saves the data in single location.\\
Thus, the reason for us using EUDAQ is quite obvious. It makes it very easy to have several different devices in one set-up and combine them to an event based data stream. \\
Being independent software, EUDAQ still uses the pXar-core library to operate the telescope.
\begin{figure}[ht]
	\includegraphics[width=0.95\textwidth]{eudaq}
	\caption{Schematic of the EUDAQ architecture \cite{eudaq}.}
	\label{p16}
\end{figure}
% ========================================================
% TO GET IT COMPILED
% ========================================================
\chapter{List of Acronyms}
\input{acros}
\bibliographystyle{plain}
\bibliography{refs}
\end{document}