% BEGIN
% ETH STYLE -> DON'T CHANGE
\documentclass[british,11pt,a4paper]{memoir}
\usepackage[utf8]{inputenc}
\usepackage[OT1]{fontenc}
\usepackage{babel}
\usepackage[sc]{mathpazo}
\usepackage{amsmath,amssymb,amsfonts,mathrsfs}
\usepackage[amsmath,thmmarks]{ntheorem}
% =======================================================
\usepackage{soul}
\usepackage{pdfpages}
\graphicspath{ {Pics/} }
\input{template/extrapackages}
\def\labelitemii{\textopenbullet}  % sets the symbols in the itemize environment
\def\labelitemiii{$\triangleright$}
\newcommand{\no}{\noindent}
\newcommand{\as}{\\[14pt]}
\newcommand{\s}{\\[7pt]}
\newcommand{\ka}{\hspace*{0.5cm}}
\newcommand{\ma}{\hspace*{1cm}}
\newcommand{\ga}{\hspace*{1.5cm}}
\newcommand{\li}{\left|}
\newcommand{\re}{\right|}
\newcommand{\lii}{\left\langle}
\newcommand{\ree}{\right\rangle}
\newcommand{\lka}{\left(}
\newcommand{\rkz}{\right)}
\newcommand{\intsum}{\ensuremath{\int\hspace{-17pt}\sum}}
\newcommand{\intsumm}{\ensuremath{{\int}\hspace{-12pt}\sum}}
\newcommand{\const}{\text{const.}}
\newcommand{\z}{\text}
\newcommand{\h}{\hslash}
\newcommand{\ar}{\autoref}
\newcommand{\fa}{\hspace{-4pt}\downarrow}
\newcommand{\wf}{\hspace{-4pt}\uparrow}
\newcommand{\cc}{\cdot}
\newcommand{\eps}{\upvarepsilon}
\newcommand{\lagr}{\mathcal{L}}
\newcommand{\lagri}{\mathcal{L}\z{I}}
\newcommand{\lagrii}{\mathcal{L}\z{II}}
\newcommand{\ham}{\mathcal{H}}
\newcommand{\bul}{\item[\textopenbullet]}
\newcommand{\terminal}[1]{\colorbox{black}{\textcolor{white}{{\fontfamily{phv}\selectfont \large{#1}}}}}
\input{template/layoutsetup}
\input{template/theoremsetup}
\input{template/macrosetup}
\usepackage[linkcolor=black,colorlinks=true,citecolor=black,filecolor=black]{hyperref}
\providecommand\subfigureautorefname{Figure}
\newsubfloat{figure}
\makeindex
% END
\begin{document}
\section{Applied Software}\label{ssoft}
The master branch of pXar and the branch that used during the thesis can be downloaded from git: 
\begin{itemize}
	\item \url{https://github.com/psi46/pxar}
	\item \url{https://github.com/michareichmann/pxar}
\end{itemize}
More information as well a detailed installation instruction can be found at the \ac{CMS} twiki: 
\begin{itemize}
	\item \url{https://twiki.cern.ch/twiki/bin/viewauth/CMS/Pxar} $rightarrow$ branch eth-2.0
\end{itemize}
The version of EUDAQ that was used can be found here: 
\begin{itemize}
	\item \url{https://github.com/veloxid/eudaq-drs4}. 
\end{itemize}


\section{Example Configuration Files}\label{sconfig}
\subsubsection{dacParameters\_C1.dat}
\ubuntu
\begin{tabular}{llr}
\hline
1	&	vdig	&	6	\\
2	&	vana	&	119	\\
3	&	vsh	&	140	\\
4	&	vcomp	&	10	\\
5	&	vleak	&	0	\\
6	&	vrgpr	&	0	\\
7	&	vwllpr	&	35	\\
8	&	vrgsh	&	0	\\
9	&	vwllsh	&	35	\\
10	&	vhlddel	&	160	\\
11	&	vtrim	&	100	\\
12	&	vthrcomp	&	101	\\
13	&	vibias\_bus	&	30	\\
14	&	vbias\_sf	&	10	\\
15	&	voffsetop	&	30	\\
16	&	vibiasop	&	50	\\
17	&	phoffset	&	120	\\
18	&	vion	&	130	\\
19	&	vcomp\_adc	&	120	\\
20	&	phscale	&	109	\\
21	&	vibias\_roc	&	150	\\
22	&	vicolor	&	99	\\
23	&	vnpix	&	40	\\
24	&	vsumcol	&	6	\\
25	&	vcal	&	255	\\
26	&	caldel	&	140	\\
253	&	ctrlreg	&	4	\\
254	&	wbc	&	100	\\\hline
\end{tabular}

\subsubsection{tbParameters.dat}
\ubuntu
\begin{tabular}{llr}
\hline
0	&	clk	&	12	\\
1	&	ctr	&	12	\\
2	&	sda	&	24	\\
3	&	tin	&	14	\\
4	&	tout	&	5	\\
252	&	level	&	15	\\
247	&	tindelay	&	14	\\
248	&	toutdelay	&	8	\\
251	&	triggerlatency	&	80	\\
254	&	deser160phase	&	4	\\\hline
\end{tabular}

\subsubsection{configParameters.dat}
\ubuntu
\begin{tabular}{llr}
\hline
\multicolumn{2}{l}{-- parameter files}		\\
	&				\\
tbParameters	&	tbParameters.dat			\\
dacParameters	&	dacParameters			\\
tbmParameters	&	tbmParameters			\\
trimParameters	&	trimParameters			\\
testParameters	&	testParameters.dat			\\
maskFile	&	defaultMaskFile.dat			\\
rootFileName	&	pxar.root			\\
	&				\\
\multicolumn{2}{l}{-- configuration}	\\
	&				\\
nModules	&	1			\\
\#nRocs	&	1	i2c:	0,1,2,3,4,5,6,7,8,9,10,11,12,13,14,15	\\
nRocs	&	1	i2c:	1	\\
nTbms	&	0			\\
hubId	&	0			\\
tbmEnable	&	0			\\
tbmEmulator	&	0			\\
hvOn	&	1			\\
tbmChannel	&	0			\\
rocType	&	psi46v2			\\
	&				\\
probeA1	&	sdata1			\\
probeA2	&	sdata2			\\
probeD1	&	clk			\\
probeD2	&	ctr			\\
	&				\\
\multicolumn{2}{l}{-- voltages and current}	\\
	&				\\
ia	&	1.199			\\
id	&	1			\\
va	&	1.699			\\
vd	&	2.5			\\

\hline
\end{tabular}

\subsubsection{trimParameters.dat}
\ubuntu
\begin{tabular}{lllr}
\hline
3	&	Pix	&	0	&	0	\\
2	&	Pix	&	0	&	1	\\
5	&	Pix	&	0	&	2	\\
5	&	Pix	&	0	&	3	\\
4	&	Pix	&	0	&	4	\\
4	&	Pix	&	0	&	5	\\
4	&	Pix	&	0	&	6	\\
5	&	Pix	&	0	&	7	\\
4	&	Pix	&	0	&	8	\\
8	&	Pix	&	0	&	9	\\
8	&	Pix	&	0	&	10	\\
\multicolumn{4}{c}{$\vdots$}\\
\hline
\end{tabular}
% ========================================================
% TO GET IT COMPILED
% ========================================================
\chapter{Acronyms}
\input{acros}
\bibliographystyle{plain}
% \bibliography{refs}
\end{document}

